\documentclass[english]{../thermomemo/thermomemo}
\usepackage[utf8]{inputenc}
\usepackage{amsmath}
\usepackage{array}% improves tabular environment.
\usepackage{dcolumn}% also improves tabular environment, with decimal centring.
\usepackage{booktabs}
\usepackage{todonotes}
\usepackage{subcaption,caption}
\usepackage{xspace}
\usepackage{tikz}
\usetikzlibrary{arrows}
\usetikzlibrary{snakes}
\usepackage{verbatim}
\usepackage{xcolor}
\hypersetup{
  colorlinks,
  linkcolor={red!50!black},
  citecolor={blue!50!black},
  urlcolor={blue!80!black}
}

%
% Egendefinerte
%
% Kolonnetyper for array.sty:
\newcolumntype{C}{>{$}c<{$}}% for å slippe å taste inn disse $
\newcolumntype{L}{>{$}l<{$}}% for å slippe å taste inn disse $
%
\newcommand*{\unit}[1]{\ensuremath{\,\mathrm{#1}}}
\newcommand*{\uunit}[1]{\ensuremath{\mathrm{#1}}}
%\newcommand*{\od}[3][]{\frac{\mathrm{d}^{#1}#2}{\mathrm{d}{#3}^{#1}}}% ordinary derivative
\newcommand*{\od}[3][]{\frac{\dif^{#1}#2}{\dif{#3}^{#1}}}% ordinary derivative
\newcommand*{\pd}[3][]{\frac{\partial^{#1}#2}{\partial{#3}^{#1}}}% partial derivative
\newcommand*{\pdc}[3]{\frac{\partial^{2}#1}{\partial{#2}\partial{#3}}}% partial derivative
\newcommand*{\pdt}[3][]{{\partial^{#1}#2}/{\partial{#3}^{#1}}}% partial
                                % derivative for inline use.
\newcommand{\pone}[3]{\frac{\partial #1}{\partial #2}_{#3}}% partial
                                % derivative with information of
                                % constant variables
\newcommand{\ponel}[3]{\frac{\partial #1}{\partial #2}\bigg|_{#3}} % partial derivative with informatio of constant variable. A line is added.
\newcommand{\ptwo}[3]{\frac{\partial^{2} #1}{\partial #2 \partial
    #3}} % partial differential in two different variables
\newcommand{\pdn}[3]{\frac{\partial^{#1}#2}{\partial{#3}^{#1}}}% partial derivative

% Total derivative:
\newcommand*{\ttd}[2]{\frac{\mathrm{D} #1}{\mathrm{D} #2}}
\newcommand*{\td}[2]{\frac{\mathrm{d} #1}{\mathrm{d} #2}}
\newcommand*{\ddt}{\frac{\partial}{\partial t}}
\newcommand*{\ddx}{\frac{\partial}{\partial x}}
% Vectors etc:
% For Computer Modern:

\DeclareMathAlphabet{\mathsfsl}{OT1}{cmss}{m}{sl}
\renewcommand*{\vec}[1]{\boldsymbol{#1}}%
\newcommand*{\vektor}[1]{\boldsymbol{#1}}%
\newcommand*{\tensor}[1]{\mathsfsl{#1}}% 2. order tensor
\newcommand*{\matr}[1]{\tensor{#1}}% matrix
\renewcommand*{\div}{\boldsymbol{\nabla\cdot}}% divergence
\newcommand*{\grad}{\boldsymbol{\nabla}}% gradient
% fancy differential from Claudio Beccari, TUGboat:
% adjusts spacing automatically
\makeatletter
\newcommand*{\dif}{\@ifnextchar^{\DIfF}{\DIfF^{}}}
\def\DIfF^#1{\mathop{\mathrm{\mathstrut d}}\nolimits^{#1}\gobblesp@ce}
\def\gobblesp@ce{\futurelet\diffarg\opsp@ce}
\def\opsp@ce{%
  \let\DiffSpace\!%
  \ifx\diffarg(%
    \let\DiffSpace\relax
  \else
    \ifx\diffarg[%
      \let\DiffSpace\relax
    \else
      \ifx\diffarg\{%
        \let\DiffSpace\relax
      \fi\fi\fi\DiffSpace}
\makeatother
%
\newcommand*{\me}{\mathrm{e}}% e is not a variable (2.718281828...)
%\newcommand*{\mi}{\mathrm{i}}%  nor i (\sqrt{-1})
\newcommand*{\mpi}{\uppi}% nor pi (3.141592...) (works for for Lucida)
%
% lav tekst-indeks/subscript/pedex
\newcommand*{\ped}[1]{\ensuremath{_{\text{#1}}}}
% høy tekst-indeks/superscript/apex
\newcommand*{\ap}[1]{\ensuremath{^{\text{#1}}}}
\newcommand*{\apr}[1]{\ensuremath{^{\mathrm{#1}}}}
\newcommand*{\pedr}[1]{\ensuremath{_{\mathrm{#1}}}}
%
\newcommand*{\volfrac}{\alpha}% volume fraction
\newcommand*{\surften}{\sigma}% coeff. of surface tension
\newcommand*{\curv}{\kappa}% curvature
\newcommand*{\ls}{\phi}% level-set function
\newcommand*{\ep}{\Phi}% electric potential
\newcommand*{\perm}{\varepsilon}% electric permittivity
\newcommand*{\visc}{\mu}% molecular (dymamic) viscosity
\newcommand*{\kvisc}{\nu}% kinematic viscosity
\newcommand*{\cfl}{C}% CFL number

\newcommand*{\cons}{\vec U}
\newcommand*{\flux}{\vec F}
\newcommand*{\dens}{\rho}
\newcommand*{\svol}{\ensuremath v}
\newcommand*{\temp}{\ensuremath T}
\newcommand*{\vel}{\ensuremath u}
\newcommand*{\mom}{\dens\vel}
\newcommand*{\toten}{\ensuremath E}
\newcommand*{\inten}{\ensuremath e}
\newcommand*{\press}{\ensuremath p}
\renewcommand*{\ss}{\ensuremath a}
\newcommand*{\jac}{\matr A}
%
\newcommand*{\abs}[1]{\lvert#1\rvert}
\newcommand*{\bigabs}[1]{\bigl\lvert#1\bigr\rvert}
\newcommand*{\biggabs}[1]{\biggl\lvert#1\biggr\rvert}
\newcommand*{\norm}[1]{\lVert#1\rVert}
%
\newcommand*{\e}[1]{\times 10^{#1}}
\newcommand*{\ex}[1]{\times 10^{#1}}%shorthand -- for use e.g. in tables
\newcommand*{\exi}[1]{10^{#1}}%shorthand -- for use e.g. in tables
\newcommand*{\nondim}[1]{\ensuremath{\mathit{#1}}}% italic iflg. ISO. (???)
\newcommand*{\rey}{\nondim{Re}}
\newcommand*{\acro}[1]{\textsc{\MakeLowercase{#1}}}%acronyms etc.

\newcommand{\nto}{\ensuremath{\mbox{N}_{\mbox{\scriptsize 2}}}}
\newcommand{\chfire}{\ensuremath{\mbox{CH}_{\mbox{\scriptsize 4}}}}
%\newcommand*{\checked}{\ding{51}}
\newcommand{\coto}{\ensuremath{\text{CO}_{\text{\scriptsize 2}}}}
\newcommand{\celsius}{\ensuremath{^\circ\text{C}}}
\newcommand{\clap}{Clapeyron~}
\newcommand{\subl}{\ensuremath{\text{sub}}}
\newcommand{\spec}{\text{spec}}
\newcommand{\sat}{\text{sat}}
\newcommand{\sol}{\text{sol}}
\newcommand{\liq}{\text{liq}}
\newcommand{\vap}{\text{vap}}
\newcommand{\amb}{\text{amb}}
\newcommand{\tr}{\text{tr}}
\newcommand{\crit}{\text{crit}}
\newcommand{\entr}{\ensuremath{\text{s}}}
\newcommand{\fus}{\text{fus}}
\newcommand{\flash}[1]{\ensuremath{#1\text{-flash}}}
\newcommand{\spce}[2]{\ensuremath{#1\, #2\text{ space}}}
\newcommand{\spanwagner}{\text{Span--Wagner}}
\newcommand{\triplepoint}{\text{TP triple point}}
\newcommand{\wrpt}{\text{with respect to}\xspace}
\newcommand{\excess}{\text{E}}

\title{Wong-Sandler equation of state}
\author{Morten Hammer}

\graphicspath{{gfx/}}

\begin{document}
\frontmatter
\tableofcontents
\section{Introduction}
Derivation of the Wong-Sandler first and second differentials needed
for thermopack implementation\cite{Wong1992}. The appendix of the
Wong-Sandler paper contain parameter definitions and first order
differentials \wrpt mole numbers for the Peng-Robinson equation of state. 
\section{Infinite pressure limit of the equations of state}
To determine the $C$ parameter of equation A3 and A4, the pressure
limit of the equation of state must be found.

\begin{table}[htb]
\begin{minipage}[l]{0.58\linewidth}
\caption{Different equations of state}
\vspace{-3mm}
\begin{tabular}{| l l l |}
\hline
		&\vspace{-4mm}		&    \\
Year: 	& 	Name:	 			&         Functional form:   \\
		&\vspace{-4mm}		&    \\
\hline
		&\vspace{-4mm}		&    \\
1873	&	Van der Waals		&	$P=\frac{RT}{v-b}-\frac{a}{v^2} $ 	 \\
		&\vspace{-4mm}		&    \\

1949	&	Redlich Kwong	 	&         $P=\frac{RT}{v-b}-\frac{a/T^{0.5}}{v\left(v+b\right)}$    	 \\
		&\vspace{-4mm}		&    \\

1972	&	Soave Redlich Kwong		&	$P=\frac{RT}{v-b}-\frac{\alpha(T)a}{v\left(v+b\right)} $       	  \\
		&\vspace{-4mm}		&    \\

1976	&	Peng Robinson		&	$P=\frac{RT}{v-b}-\frac{\alpha(T)a}{v\left(v+b\right)+b(v-b)} $      	  \\

		&\vspace{-4mm}		&    \\

1980	&	Schmidt-Wensel		&	$P=\frac{RT}{v-b}-\frac{\alpha(T)a}{v^2+(1+3c)bv-3cb^2} $      \\

		&\vspace{-4mm}		&    \\

1982	&	Patel Teja			&	$P=\frac{RT}{v-b}-\frac{\alpha(T)a}{v\left(v+b\right)+c(v-b)} $        \\
		&\vspace{-4mm}		&    \\

1987	&	Yu-Lu				&	$P=\frac{RT}{v-b}-\frac{\alpha(T)a}{v\left(v+b\right)+b(3v+c)} $     \\
		&\vspace{-4mm}		&    \\

1987	&	Trebble-Bishnoi		&	$P=\frac{RT}{v-b}-\frac{\alpha(T)a}{v^2+(b(T)+c)v-b(T)c-d^2} $    \\
\hline
\end{tabular}

\label{tab:eos}
\end{minipage}
\end{table}

\begin{equation}
P=\frac{nRT}{V-B}-\frac{A}{\left(V-B(T)m_1\right)\left(V-B(T)m_2\right)}
\label{eq:eosg}
\end{equation}

\begin{table}[htb]
\begin{minipage}[l]{0.58\linewidth}
\caption{Equation of state parameters}
\vspace{-3mm}
\begin{tabular}{| l l l |}
\hline
		&\vspace{-4mm}		&    \\
	Name:	 			&        $ m_1$ 	&	$m_2$  \\
		&\vspace{-4mm}		&    \\
\hline
		&\vspace{-4mm}		&    \\
Van der Waals:		&	0	&	0	 \\
		&\vspace{-4mm}		&    \\

Redlich Kwong:	 	&        0	&	-1   	 \\
		&\vspace{-4mm}		&    \\

Soave Redlich Kwong:		&	0	&	-1     	  \\
		&\vspace{-4mm}		&    \\

Peng Robinson:		&	-1+$\sqrt{2}$		&	-1-$\sqrt{2}$    	  \\

		&\vspace{-4mm}		&    \\

Schmidt-Wensel		&	$\frac{-1-3w+\sqrt{1+18w+9w^2}}{2}$	&	$\frac{-1-3w-\sqrt{1+18w+9w^2}}{2}$	      \\

		&\vspace{-4mm}		&    \\

Patel Teja:			&	$\frac{-b-c+\sqrt{b^2+2bc+c^2+4bc}}{2b}$	&	$\frac{-b-c-\sqrt{b^2+2bc+c^2+4bc}}{2b}$        \\
		&\vspace{-4mm}		&    \\

Yu-Lu				&	$\frac{-c-3b+\sqrt{c^2+2bc+9b^2}}{2b}$	&	$\frac{-c-3b-\sqrt{c^2+2bc+9b^2}}{2b}$	     \\
		&\vspace{-4mm}		&    \\

Trebble-Bishnoi		&	$\frac{-b(T)-c+\sqrt{b(T)^2+6b(T)c+c^2+4d}}{2b(T)}$	&  $\frac{-b(T)-c-\sqrt{b(T)^2+6b(T)c+c^2+4d}}{2b(T)}$		   \\
\hline
\end{tabular}

\label{tab:eos2}
\end{minipage}
\end{table}

The residual Helmholtz function, $F$, becomes
\begin{equation}
  F(T,V,\vektor{n})=
  n\left[\ln\left(\frac{V}{V-B}\right)-\frac{A}{\left(m_1-m_2\right)BRT}\ln\left(\frac{V-m_2B}{V-m_1B}\right) \right].
\end{equation}

In order to find the pressure limit, we use that,
\begin{equation}
  F(T,P,\vektor{n}) = F(T,V,\vektor{n}) - \ln(Z),
\end{equation}
where $Z=PV/(nRT)$. We then get,
\begin{equation}
  F(T,P,\vektor{n})=
  n\left[-\ln\left(\frac{P\left(V-B\right)}{nRT}\right)-\frac{A}{\left(m_1-m_2\right)BRT}\ln\left(\frac{V-m_2B}{V-m_1B}\right)
  \right].
\end{equation} 
From \ref{eq:eosg} we see
\begin{equation}
\lim_{P\to\infty} \frac{P\left(V-B\right)}{nRT} = 1
\label{eq:plim}
\end{equation}
We therefore get
\begin{equation}
  \lim_{P\to\infty} F(T,P,\vektor{n})=
  \frac{nA}{BRT}
  \frac{1}{\left(m_1-m_2\right)}\ln\left(\frac{B-m_1B}{B-m_2B}\right)
  = \frac{nAC}{BRT},
  \label{eq:plimF}
\end{equation}
where $C$ is given as follows,
\begin{equation}
  C=
  \frac{1}{\left(m_1-m_2\right)}\ln\left(\frac{1-m_1}{1-m_2}\right).
  \label{eq:C}
\end{equation}
 
\begin{table}[htb]
\caption{Equation of state $C$-parameter}
\center
\begin{tabular}{ll}
  \hline
  EOS &         $C$   \\
  \hline
  Van der Waals		& -1 \\
  Redlich Kwong	 	& -$\ln{2}$ \\
  Soave Redlich Kwong	& -$\ln{2}$ \\
  Peng Robinson	        &
  $\frac{1}{\sqrt{2}}\ln{\left(\sqrt{2}-1\right)}$ \\
  Otherwise             & Use Eq. \ref{eq:C} \\
  \hline
\end{tabular}
\label{tab:C}
\end{table}
%\todo[inline]{Calculate $C$ for the three parameter equations of state}

\section{The Wong-Sandler mixing rule}
The Wong-Sandler mixing rule give a relation between the $B$ and $A$
parameter of the equation of state. In effect the $B$ parameter become
temperature dependent.

\begin{align}
  B &= \frac{Q}{n-D}   \label{eq:bm}\\
  &\Updownarrow \nonumber\\
  B(n-D) &= Q\label{eq:bm_diff}
\end{align}

\begin{align}
  \frac{A}{RT} &= BD
  \label{eq:am}\\
  &\Updownarrow \nonumber\\
  A &= BDRT
  \label{eq:am_diff}
\end{align}
$Q$ is defined as follows,
\begin{equation}
  Q = \underset{i}{\sum}\underset{j}{\sum}n_in_j\left(b-\frac{a}{RT}\right)_{ij},   \label{eq:Q}
\end{equation}
where
\begin{equation}
\left(b-\frac{a}{RT}\right)_{ij} =\frac{\left(b_i-\frac{a_i(T)}{RT}\right) + \left(b_j-\frac{a_j(T)}{RT}\right)}{2}\left(1-k_{ij}\left(T\right)\right),
\end{equation}
and
\begin{align}
%   b_{ij} &= \frac{b_i+b_j}{2},   \label{eq:bij}\\
%   a_{ij} &= \sqrt{a_ia_j}\left(1-k_{ij}\right),   \label{eq:aij}
   a_{i}(T) &= \tilde{a}_i \alpha(T),   \label{eq:aai}
\end{align}
To simplify the differentials, we introduce $r_i$,
\begin{align}
  r_i&=\left(b_i-\frac{a_i(T)}{RT}\right),\\
  r_{i,T}&=-\frac{a_{i,T}}{RT} + \frac{a_i}{RT^2},\\
  r_{i,TT}&=-\frac{a_{i,TT}}{RT} + \frac{2a_{i,T}}{RT^2} - \frac{2a_{i}}{RT^3}.\\
\end{align}
$D$ is defined as,
\begin{equation}
  D = \underset{i}{\sum}n_i\frac{a_i}{b_iRT} + \frac{A_\infty^\excess}{CRT} \label{eq:D}
\end{equation}
\section{Partial differentials}
\subsection{Temperature differentials}
Differentiating \ref{eq:bm_diff} \wrpt $T$ gives:
\begin{align}
  B_T(n-D)-BD_T &= Q_T   \label{eq:bmT_diff}\\
  &\Updownarrow \nonumber\\
  B_T &=\frac{Q_T+BD_T}{n-D} \label{eq:bmT}
\end{align}
Further differentiating \ref{eq:bmT_diff} \wrpt $T$ gives:
\begin{align}
  B_{TT}(n-D)-2B_TD_T-BD_{TT} &= Q_{TT}   \label{eq:bmTT_diff}\\
  &\Updownarrow \nonumber\\
  B_{TT} &=\frac{Q_{TT}+2B_TD_T+BD_{TT}}{n-D} \label{eq:bmTT}
\end{align}
Differentiating \ref{eq:am_diff} \wrpt $T$ gives:
\begin{align}
  A_T &= B_TDRT + BD_TRT + BDR   \label{eq:amT_diff}\\
\end{align}
Further differentiating \ref{eq:amT_diff} \wrpt $T$ gives:
\begin{align}
  A_{TT} &= B_{TT}DRT + 2B_{T}D_TRT + 2B_{T}DR + 2BD_TR + BD_{TT}RT
  \label{eq:amTT_diff}
\end{align}
Differentiating \ref{eq:Q}, gives,
\begin{eqnarray}
  Q_T =&\underset{i}{\sum}\underset{j}{\sum}n_in_j&\left(\frac{\left(r_{i,T}+r_{j,T}\right)\left(1-k_{ij}\right)}{2}-\frac{r_i+r_j}{2}k_{ij,T} \right), \label{eq:QT}\\
  Q_{TT} =&\underset{i}{\sum}\underset{j}{\sum}n_in_j&\bigg(\frac{\left(r_{i,TT}+r_{j,TT}\right)\left(1-k_{ij}\right)}{2}-\left(r_{i,T}+r_{j,T}\right)k_{ij,T} \\
& & -\frac{r_i+r_j}{2}k_{ij,TT} \bigg),  \label{eq:QTT}
\end{eqnarray}
Differentiating \ref{eq:D}, gives,
\begin{align}
  D_TRT + DR &= \underset{i}{\sum}n_i\frac{a_{i,T}}{b_i} + \frac{A_{\infty,T}^\excess}{C},   \label{eq:DT_diff}\\
  &\Updownarrow \nonumber\\
  D_T &= \frac{\underset{i}{\sum}n_i\frac{a_{i,T}}{b_i} + \frac{A_{\infty,T}^\excess}{C} - DR}{RT}.   \label{eq:DT}
\end{align}
Differentiating \ref{eq:DT_diff} further gives,
\begin{align}
  D_{TT}RT + 2D_TR &= \underset{i}{\sum}n_i\frac{a_{i,TT}}{b_i} + \frac{A_{\infty,TT}^\excess}{C},   \label{eq:DTT_diff}\\
  &\Updownarrow \nonumber\\
  D_{TT} &= \frac{\underset{i}{\sum}n_i\frac{a_{i,TT}}{b_i} + \frac{A_{\infty,TT}^\excess}{C} - 2D_TR}{RT}.   \label{eq:DTT}
\end{align}
\subsection{Mole number differentials}
Differentiating \ref{eq:bm_diff} \wrpt $n_i$ gives:
\begin{align}
  B_i(n-D)+B(1-D_i) &= Q_i   \label{eq:bmi_diff}\\
  &\Updownarrow \nonumber\\
  B_i &=\frac{Q_i+B(D_i-1)}{n-D} \label{eq:bmi}
\end{align}
Further differentiating \ref{eq:bmi_diff} \wrpt $n_j$ gives:
\begin{align}
  B_{ij}(n-D)+B_i(1-D_j)+B_j(1-D_i)-BD_{ij} &= Q_{ij}   \label{eq:bmij_diff}\\
  &\Updownarrow \nonumber\\
  \frac{Q_{ij}+B_i(D_j-1)+B_j(D_i-1)+BD_{ij}}{n-D} &= B_{ij} \label{eq:bmij}
\end{align}
Differentiating \ref{eq:am_diff} \wrpt $n_i$ gives:
\begin{align}
  A_i &= RT\left(B_iD + BD_i\right)    \label{eq:ami_diff}\\
\end{align}
Further differentiating \ref{eq:ami_diff} \wrpt $n_j$ gives:
\begin{align}
  A_{ij} &= RT\left(B_{ij}D + B_{i}D_j + B_{j}D_i + BD_{ij}\right)
  \label{eq:amij_diff}
\end{align}
Differentiating \ref{eq:Q}, \wrpt $n_i$ and thereafter $n_j$ gives,
\begin{align}
  Q_i &=2\underset{j}{\sum}n_j\left(b-\frac{a}{RT}\right)_{ij}, \label{eq:Qi}\\
  Q_{ij} &= 2\left(b-\frac{a}{RT}\right)_{ij}.   \label{eq:Qij}
\end{align}

Differentiating Equation \ref{eq:D} \wrpt $n_i$ gives,
\begin{align}
  D_iRT &= \frac{a_{i}}{b_i} + \frac{A_{\infty,i}^\excess}{C}.   \label{eq:Di_diff}
\end{align}
Differentiating \ref{eq:Di_diff} further \wrpt $n_j$ gives,
\begin{align}
  D_{ij}RT &=  \frac{A_{\infty,ij}^\excess}{C}.   \label{eq:Dij_diff}
\end{align}
\subsection{Cross differentials}
Differentiating \ref{eq:bmi_diff} \wrpt $T$ gives:
\begin{align}
  B_{iT}(n-D)-B_iD_T+B_T(1-D_i)-BD_{iT} &= Q_{iT}   \label{eq:bmit_diff}\\
  &\Updownarrow \nonumber\\
  \frac{Q_{iT}+B_iD_T+B_T(D_i-1)+BD_{iT}}{n-D} &= B_{iT} \label{eq:bmit}
\end{align}
Differentiating \ref{eq:ami_diff} \wrpt $T$ gives:
\begin{align}
  A_{iT} &= R\left(B_{iT}DT + B_{i}D_TT + B_{i}D + B_TD_iT + BD_{iT}T + BD_i\right)   \label{eq:amit_diff}\\
\end{align}
Differentiating \ref{eq:Qi}, \wrpt $T$ gives,
\begin{align}
  %Q_{iT} &=2\underset{i}{\sum}n_i\left(b-\frac{a}{RT}\right)_{ij}, \label{eq:Qit}
  Q_{iT} =&2\underset{j}{\sum}n_j\left(\frac{\left(r_{i,T}+r_{j,T}\right)\left(1-k_{ij}\right)}{2}-\frac{r_i+r_j}{2}k_{ij,T} \right). \label{eq:QiT}\\
\end{align}

Differentiating Equation \ref{eq:Di_diff} \wrpt $T$ gives,
\begin{align}
  D_{iT}RT + D_iR &= \frac{a_{i,T}}{b_i} + \frac{A_{\infty,iT}^\excess}{C}.   \label{eq:Dit_diff}\\
  &\Updownarrow \nonumber\\
  D_{iT} &= \frac{\frac{a_{i,T}}{b_i} + \frac{A_{\infty,iT}^\excess}{C} - D_iR}{RT}.   \label{eq:Dit}
\end{align}

\section{The NRTL infinite free energy model}
The NRTL model \cite{Renon1968}:

\begin{equation}
  \frac{A_\infty^\excess}{RT} =
  \underset{i}{\sum}n_i\left(\frac{\underset{j}{\sum}n_j\tau_{ji}g_{ji}}{\underset{k}{\sum}n_k
      g_{ki}}\right),
  \label{eq:nrtl}
\end{equation}
where
\begin{equation}
  g_{ij}\left(T\right) =\exp{\left(-\alpha_{ij}\tau_{ij}\left(T\right)\right)}.
  \label{eq:gij}
\end{equation}
Looking at the NRTL model it is very similar to the Huron-Vidal model
(HV) \cite{Huron1979},
\begin{equation}
  \frac{G_\infty^\excess}{RT} =
  \underset{i}{\sum}n_i\left(\frac{\underset{j}{\sum}n_jb_{j}\tau_{ji}g_{ji}}{\underset{k}{\sum}n_kb_{k}
      g_{ki}}\right).
  \label{eq:HV}
\end{equation}
Comparing equation \ref{eq:HV} and equation \ref{eq:nrtl}, the
substitution $b_{j}g_{ji}^{\text{HV}} = g_{ji}^{\text{NRTL}}$ can be
made, that lets us reuse the existing HV implementation. For the
combination of NRTL for some binaries and simple van der Waals mixing
for other binaries, the following can be done,
\begin{eqnarray}
g_{ji} & = & b_{j},\\
\beta_{ii} & = & \frac{a_{i}}{b_{i}}C,\\
\beta_{ji} & = & -\frac{\sqrt{b_i b_j}}{b_{ij}}\sqrt{\beta_{ii}\beta_{jj}}\left(1-k_{ij}\right),\\
\tau_{ji} &=& \frac{\beta_{ji}-\beta_{ii}}{RT}.
\end{eqnarray}
This follows from the relations for HV described in
\cite{Wilhelmsen2013}.

Alternatively the relations developed by \cite{Twu2001}, can be used.
\subsection{Differentials}
Follows the approach above, the differentials are already available
from \cite{Wilhelmsen2013}.
% Differentating \ref{eq:nrtl}
% \begin{align}
%   A_{\infty,T}^\excess =& RA_{\infty}^\excess + 
%   RT\underset{i}{\sum}n_i\left(\frac{\underset{j}{\sum}n_j\tau_{ji,T}g_{ji}\left(1-\alpha_{ij}\tau_{ij}\right)}{\underset{k}{\sum}n_k
%       g_{ki}}\right) \\
%   &+
% RT\underset{i}{\sum}n_i\left(\frac{\underset{j}{\sum}n_j\tau_{ji}g_{ji}}{\left(\underset{k}{\sum}n_k
%       g_{ki}\right)^2}\underset{k}{\sum}-n_k
%       g_{ki}\alpha_{ki}\tau_{ij,T}\right).
%   \label{eq:nrtl_T}
% \end{align}
% Further differentating \ref{eq:nrtl_T},
% \begin{equation}
%   A_{\infty,TT}^\excess = RA_{\infty,T}^\excess + 
%   RT\underset{i}{\sum}n_i\left(\frac{\underset{j}{\sum}n_j\tau_{ji}g_{ji}}{\underset{k}{\sum}n_k
%       g_{ki}}\right),
%   \label{eq:nrtl_TT}
% \end{equation}
%\section{The UNIFAC model}
%The Wong-Sandler can be used with the UNIFAC model of \cite{}.
% Need expression for the residual excess gibbs energy
\clearpage
\bibliographystyle{plain}
\bibliography{../thermopack}

\end{document}
