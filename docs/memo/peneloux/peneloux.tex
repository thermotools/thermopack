\documentclass[english]{../thermomemo/thermomemo}
\usepackage[utf8]{inputenc}

\title{Volume shift for generic EOS}
\author{Morten Hammer}

\usepackage[normalem]{ulem}

\usepackage{hyperref}
\usepackage{color}

\definecolor{midnightblue}{RGB}{35,35,132}
\definecolor{urlblue}{RGB}{70,130,180}

\definecolor{shadecolor}{gray}{0.9}

\hypersetup{
    colorlinks=true,
    linkcolor=midnightblue,
    urlcolor=urlblue,
    citecolor=midnightblue,
    linktoc=page
}

\usepackage{amsmath}
\usepackage{hyperref}
\usepackage{framed}
\usepackage{siunitx,mhchem,todonotes}
\usepackage{xspace}
\newcommand{\pone}[3]{\frac{\partial #1}{\partial #2}\bigg|_{#3}}% partial
                                % derivative with information of
                                % constant variables
\newcommand*{\vektor}[1]{\boldsymbol{#1}}%
\newcommand{\dd}[1]{\mathrm{d}{#1}}
\newcommand{\eos}{\ensuremath{\text{eos}}\xspace}

\DeclareMathOperator*{\argmin}{arg\,min }



\usepackage[activate={true,nocompatibility},final,kerning=true,tracking=true,spacing=true,stretch=10,shrink=10]{microtype}
\microtypecontext{spacing=nonfrench}
\SetExtraKerning[unit=space]
    {encoding={*}, family={bch}, series={*}, size={footnotesize,small,normalsize}}
    {\textendash={400,400}, % en-dash, add more space around it
     "28={ ,150}, % left bracket, add space from right
     "29={150, }, % right bracket, add space from left
     \textquotedblleft={ ,150}, % left quotation mark, space from right
     \textquotedblright={150, }} % right quotation mark, space from left
\SetTracking{encoding={*}, shape=sc}{0}

\begin{document}
\frontmatter

%\tableofcontents

\section{Introduction}
The volume shift was introduced by P{\'e}neloux et al. \cite{Peneloux1982},
\begin{equation}
  c = \frac{1}{n}\underset{i}{\sum}c_i n_i,
\label{eq:volumeshift}
\end{equation}
where $c_i$ is a component constant representing the component volume
shift.

Different properties change when working with volume translations, see
Jaubert et al. \cite{Jaubert2016} for details.

The volume-shift have found application in many cubic based equations
of state (t-mPR\cite{Kordas1995}, PSRK\cite{Fischer1996},
VTPR\cite{Collinet2006}, tc-PR/tc-RK\cite{LeGuennec2016}, \dots), and
the component volume translations $c_i$, are often fixated to match the
liquid density at $T=0.7T_{\text{Crit}}$,
\section{Volume shifts for generic EOS}

The residual reduced Helmholtz function of a  generic EOS is found as follows,
\begin{align}
  F(T,V_{\eos},\vektor{n}) = \frac{A^\text{r}(T,V_{\eos},\vektor{n})}{RT}
  = \int^\infty_{V_{\eos}} \left[ \frac{P(T,V_{\eos}^\prime,\vektor{n})}{RT} - \frac{n}{V_{\eos}^\prime} \right]\dd{V_{\eos}^\prime}
  \label{eq:helmholtz_int_eos}
\end{align}
Introducing the volume shift,
\begin{equation}
V = V_{\eos}- \sum n_ic_i = V_{\eos}- C,
\label{eq:v_shift}
\end{equation}
The residual reduced helmholtz of the volume-shifted (vs) EOS can be
found, using $dV = dV_{\eos}$ at constant $n$ and $T$,
\begin{align}
  F^{\text{vs}}(T,V,\vektor{n})
  &= \int^\infty_V \left[ \frac{P(T,V^\prime+C,\vektor{n})}{RT} - \frac{n}{V^\prime} \right]\dd{V^\prime} \\ &= \int^\infty_V \left[ \frac{P(T,V^\prime+C,\vektor{n})}{RT} - \frac{n}{V^\prime + C} \right]\dd{V^\prime} + n\int^\infty_V \left[\frac{1}{V^\prime + C}  - \frac{1}{V^\prime} \right]\dd{V^\prime}\\ &= \int^\infty_{V_{\eos}} \left[ \frac{P(T,V_{\eos}^\prime,\vektor{n})}{RT} - \frac{n}{V_{\eos}^\prime} \right]\dd{V_{\eos}^\prime} + n\int^\infty_V \left[\frac{1}{V^\prime + C}  - \frac{1}{V^\prime} \right]\dd{V^\prime}\\ &= F(T,V_{\eos},\vektor{n})  + n \ln \left(\frac{V}{V_{\eos}} \right)
  \label{eq:helmholtz_int}
\end{align}
Here we need to treat $V_{\eos} = V_{\eos}(V,\vektor{n})$ with the
chain rule when differentiating $F^{\text{vs}}$.

If we introduce $F^C$ as the corrected residual reduced Helmholtz energy, due to
the difference in ideal volume,
\begin{align}
  F^C(V,\vektor{n}) &= n \ln \left(\frac{V}{V+C} \right),
  \label{eq:F_corr}
\end{align}
the differentials can be derived in a organized manner.
\begin{align}
  F^C_{V} &= n \left(\frac{1}{V} - \frac{1}{V+C} \right) = n \left(\frac{1}{V} - \frac{1}{V_{\eos}} \right), \\
  F^C_{VV} &= n \left(-\frac{1}{V^2} + \frac{1}{\left(V+C\right)^2} \right) = n \left(-\frac{1}{V^2} + \frac{1}{V_{\eos}^2} \right), \\
  F^C_{i} &= \ln \left(\frac{V}{V+C} \right) - \frac{nc_i}{V+C} = \ln \left(\frac{V}{V_{\eos}} \right) - \frac{nc_i}{V_{\eos}}, \\
  F^C_{ij} &= -\frac{\left(c_j + c_i \right)}{V+C} +  \frac{n c_ic_j}{\left(V+C\right)^2} = -\frac{\left(c_j + c_i \right)}{V_{\eos}} +  \frac{n c_ic_j}{V_{\eos}^2}, \\
  F^C_{Vi} &= \frac{1}{V} - \frac{1}{V+C} + \frac{nc_i}{\left(V+C\right)^2} =  \frac{1}{V} - \frac{1}{V_{\eos}} + \frac{nc_i}{V_{\eos}^2}
\end{align}
In addition the compositional differentials change since $V_{\eos} = V + C$,
\begin{align}
  F^{\eos}_{i} &= F^{\eos}_{i} + F^{\eos}_{V_{\eos}} c_i , \label{eq:Fi}\\
  F^{\eos}_{Ti} &= F^{\eos}_{Ti} + F^{\eos}_{TV_{\eos}} c_i , \\
  F^{\eos}_{ij} &= F^{\eos}_{ij} + F^{\eos}_{iV_{\eos}} c_j + F^{\eos}_{V_{\eos}j} c_i + F^{\eos}_{V_{\eos}V_{\eos}} c_ic_j . \label{eq:Fij}
\end{align}

\subsection{Test of the fugacity coefficient}
Let us test this for the fugacity coefficient. It is defined as
\begin{align}
  \ln \hat{\varphi}_i^{\text{vs}} = \biggl( \frac{\partial F^{\text{vs}}}{\partial n_i} \biggr)_{T,V,n_j} - \ln \left( Z \right) = F_{n_i}^{\text{vs}} - \ln \left( Z \right)
  \label{eq:fugacity}
\end{align}
Differentiating $F^{\text{vs}}$,
\begin{align}
  F_{n_i}^{\text{vs}} &= F_{n_i} + F_{V_{\eos}}c_i  + \ln \left(\frac{V}{V_{\eos}} \right) - \frac{n c_i}{V_{\eos}} = F_{n_i} + \ln \left(\frac{V}{V_{\eos}} \right) - \frac{P c_i}{RT}
  \label{eq:Fni}
\end{align}
Combining Equation \ref{eq:fugacity} and \ref{eq:Fni}, we get
\begin{align}
  \ln \hat{\varphi}_i^{\text{vs}} &= F_{n_i} + \ln \left(\frac{V}{V_{\eos}} \right) - \frac{P c_i}{RT} - \ln \left( \frac{PV}{n RT} \right) \\
                      &= F_{n_i} - \ln \left( \frac{PV_{\eos}}{n RT} \right) - \frac{P c_i}{RT} \\
  &= \ln \hat{\varphi}_i - \frac{P c_i}{RT}
  \label{eq:fugacity2}
\end{align}
which is the same result as reported by P{\'e}neloux et al.


\section{Correlations used for $c_i$}
The $c_i$ for the SRK EOS is calculated from the following equation:
\begin{equation}
  c_i = 0.40768\frac{R T_{c_i}}{P_{c_i}}\left(0.29441- Z_{\text{RA}}\right)
\label{eq:ci}
\end{equation}

$Z_{\text{RA}}$ are tabulated in TPlib. Reid et al. \cite{Reid1987}
also correlate $Z_{\text{RA}}$ as follows:
\begin{equation}
  Z_{\text{RA}} = 0.29056 - 0.08775 \omega
\label{eq:zra}
\end{equation}

Jhaveri and Youngren \cite{Jhaveri1988} have developed different paramaters for the PR EOS:
\begin{equation}
  c_i^{\text{PR}} = 0.50033\frac{R T_{c_i}}{P_{c_i}}\left(0.25969- Z_{\text{RA}}\right)
\label{eq:ci_PR}
\end{equation}

\section{Temperature dependent volume shift}
Temperature dependent volume translation are known to give supercritical
iso-therm crossings \cite{Pfohl1999} and possibly un-physical
behaviour \cite{Kalikhman2010} and must be executed with care. In
some cases it can be used as a simple remedy to improve liquid density
predictions.

In this case the $F^C$ function becomes,
\begin{align}
  F^C(V,\vektor{n},T) &= n \ln \left(\frac{V}{V+C\left(\vektor{n},T\right)} \right),
  \label{eq:F_corr_t}
\end{align}
and the temperature differentials become,
\begin{align}
  F^C_{T} &= -\frac{n C_T}{V+C} = -\frac{n C_T}{V_\eos}, \\
  F^C_{TT} &= -\frac{n C_{TT}}{V+C} + \frac{n C_T^2}{\left( V+C \right)^2} = -\frac{n C_{TT}}{V_\eos} + \frac{n C_T^2}{V_\eos^2}, \\
  F^C_{VT} &= \frac{n C_T}{\left( V+C \right)^2} = \frac{n C_T}{V_\eos^2}, \\
  F^C_{iT} &= -\frac{C_T}{V+C} - \frac{n c_{iT}}{V+C} + \frac{n C_T c_i}{\left( V+C \right)^2} = -\frac{\left(C_T + n c_{iT}\right)}{V_\eos}  + \frac{n C_T c_i}{V_\eos^2}.
\end{align}
In addition the compositional and temperature differentials change
since $V_{\eos} = V + C\left(\vektor{n},T\right)$,
\begin{align}
  F^{\eos}_{T} &= F^{\eos}_{T} + F^{\eos}_{V_{\eos}} C_T , \\
  F^{\eos}_{TT} &= F^{\eos}_{TT} + 2 F^{\eos}_{TV_{\eos}} C_T + F^{\eos}_{V_{\eos}V_{\eos}} C_T^2 + F^{\eos}_{V_{\eos}} C_{TT} , \\
  F^{\eos}_{VT} &= F^{\eos}_{VT} + F^{\eos}_{V_{\eos}V_{\eos}} C_T , \\
  F^{\eos}_{Ti} &= F^{\eos}_{Ti} + F^{\eos}_{TV_{\eos}} c_i + F^{\eos}_{V_{\eos}V_{\eos}} C_T c_i + F^{\eos}_{V_{\eos}i} C_T + F^{\eos}_{V_{\eos}} c_{T,i}.
\end{align}
While $F^{\eos}_{i}$ and $F^{\eos}_{ij}$ are unchanged from
\eqref{eq:Fi} and \eqref{eq:Fij} respectively.
\clearpage
\bibliographystyle{plain}
\bibliography{../thermopack}

\end{document}

%%% Local Variables:
%%% mode: latex
%%% TeX-master: t
%%% End:
