\section{Conclusion}
PyQt has been nice to work with. First of all, Qt is well documented, its community is very helpful and solutions to many questions of all levels can be found. The Qt Designer saves a lot of time, so PyQt is a good way of creating GUI's in Python. To create such a GUI, one could as well have used C++ and gotten the same result. It would take somewhat longer time, but most of the examples on the Internet uses C++, so this seems like a more common way to create GUI's. And there are some things you can't control in the same way in Python as in C++, such as with pointers.

As of now, there are still things left to be done with the GUI. More of the widgets can be styled to create a more complete feel. In the Plot Mode, it could be added an option for plotting experimental data and compare this to the existing models. Plot Mode can also have options for changing units. Not all calculations throw an exception if the solution does not converge, and thus sometimes the application stops. This has to be fixed before distributing the application.