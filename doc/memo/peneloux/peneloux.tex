\documentclass[english]{article}
% NOTE: You must pass either norsk or english as an option!
\usepackage[utf8]{inputenc}

% \usepackage{textcomp}
% \usepackage{listings}
% \usepackage{babel}
% \usepackage{natbib}
\usepackage{amsmath}
% \usepackage{pslatex}
\usepackage{array}% improves tabular environment.
\usepackage{dcolumn}% also improves tabular environment, with decimal centring.
% \usepackage{lastpage}
% \usepackage{listings}
% \usepackage{tikz}
% \usetikzlibrary{decorations.pathmorphing}
% \usepackage{pgfplots}
% \usepackage{cleveref}
% \usepackage{pgf,pgfarrows,pgfnodes}

% \usepackage[final]{pdfpages}

% \usepackage[altbullet,expert]{lucidabr}

% \usepackage{ifthen}
% \usepackage{nomencl}
% \usepackage{parskip}
% \usepackage{psfrag}
\usepackage{booktabs}
%\usepackage{todonotes}
\usepackage{siunitx,mhchem,todonotes}
% \usepackage{pifont}
%\usepackage{subfigure} %,caption
\usepackage{subcaption,caption}
% \usepackage{microtype}% for pdflatex; medisin mot overfull \vbox

\usepackage{tikz}
\usetikzlibrary{arrows}
\usetikzlibrary{snakes}
\usepackage{verbatim}

%
% Egendefinerte
%
% Kolonnetyper for array.sty:
\newcolumntype{C}{>{$}c<{$}}% for å slippe å taste inn disse $
\newcolumntype{L}{>{$}l<{$}}% for å slippe å taste inn disse $
%
\newcommand*{\unit}[1]{\ensuremath{\,\mathrm{#1}}}
\newcommand*{\uunit}[1]{\ensuremath{\mathrm{#1}}}
%\newcommand*{\od}[3][]{\frac{\mathrm{d}^{#1}#2}{\mathrm{d}{#3}^{#1}}}% ordinary derivative
\newcommand*{\od}[3][]{\frac{\dif^{#1}#2}{\dif{#3}^{#1}}}% ordinary derivative
\newcommand*{\pd}[3][]{\frac{\partial^{#1}#2}{\partial{#3}^{#1}}}% partial derivative
\newcommand*{\pdt}[3][]{{\partial^{#1}#2}/{\partial{#3}^{#1}}}% partial
                                % derivative for inline use.
\newcommand{\pone}[3]{\frac{\partial #1}{\partial #2}_{#3}}% partial
                                % derivative with information of
                                % constant variables
\newcommand{\ponel}[3]{\frac{\partial #1}{\partial #2}\bigg|_{#3}} % partial derivative with informatio of constant variable. A line is added.
\newcommand{\ptwo}[3]{\frac{\partial^{2} #1}{\partial #2 \partial
    #3}} % partial differential in two different variables
\newcommand{\pdn}[3]{\frac{\partial^{#1}#2}{\partial{#3}^{#1}}}% partial derivative

% Total derivative:
\newcommand*{\ttd}[2]{\frac{\mathrm{D} #1}{\mathrm{D} #2}}
\newcommand*{\td}[2]{\frac{\mathrm{d} #1}{\mathrm{d} #2}}
\newcommand*{\ddt}{\frac{\partial}{\partial t}}
\newcommand*{\ddx}{\frac{\partial}{\partial x}}
% Vectors etc:
% For Computer Modern:

\DeclareMathAlphabet{\mathsfsl}{OT1}{cmss}{m}{sl}
\renewcommand*{\vec}[1]{\boldsymbol{#1}}%
\newcommand*{\vektor}[1]{\boldsymbol{#1}}%
\newcommand*{\tensor}[1]{\mathsfsl{#1}}% 2. order tensor
\newcommand*{\matr}[1]{\tensor{#1}}% matrix
\renewcommand*{\div}{\boldsymbol{\nabla\cdot}}% divergence
\newcommand*{\grad}{\boldsymbol{\nabla}}% gradient
% fancy differential from Claudio Beccari, TUGboat:
% adjusts spacing automatically
\makeatletter
\newcommand*{\dif}{\@ifnextchar^{\DIfF}{\DIfF^{}}}
\def\DIfF^#1{\mathop{\mathrm{\mathstrut d}}\nolimits^{#1}\gobblesp@ce}
\def\gobblesp@ce{\futurelet\diffarg\opsp@ce}
\def\opsp@ce{%
  \let\DiffSpace\!%
  \ifx\diffarg(%
    \let\DiffSpace\relax
  \else
    \ifx\diffarg[%
      \let\DiffSpace\relax
    \else
      \ifx\diffarg\{%
        \let\DiffSpace\relax
      \fi\fi\fi\DiffSpace}
\makeatother
%
\newcommand*{\me}{\mathrm{e}}% e is not a variable (2.718281828...)
%\newcommand*{\mi}{\mathrm{i}}%  nor i (\sqrt{-1})
\newcommand*{\mpi}{\uppi}% nor pi (3.141592...) (works for for Lucida)
%
% lav tekst-indeks/subscript/pedex
\newcommand*{\ped}[1]{\ensuremath{_{\text{#1}}}}
% høy tekst-indeks/superscript/apex
\newcommand*{\ap}[1]{\ensuremath{^{\text{#1}}}}
\newcommand*{\apr}[1]{\ensuremath{^{\mathrm{#1}}}}
\newcommand*{\pedr}[1]{\ensuremath{_{\mathrm{#1}}}}
%
\newcommand*{\volfrac}{\alpha}% volume fraction
\newcommand*{\surften}{\sigma}% coeff. of surface tension
\newcommand*{\curv}{\kappa}% curvature
\newcommand*{\ls}{\phi}% level-set function
\newcommand*{\ep}{\Phi}% electric potential
\newcommand*{\perm}{\varepsilon}% electric permittivity
\newcommand*{\visc}{\mu}% molecular (dymamic) viscosity
\newcommand*{\kvisc}{\nu}% kinematic viscosity
\newcommand*{\cfl}{C}% CFL number

\newcommand*{\cons}{\vec U}
\newcommand*{\flux}{\vec F}
\newcommand*{\dens}{\rho}
\newcommand*{\svol}{\ensuremath v}
\newcommand*{\temp}{\ensuremath T}
\newcommand*{\vel}{\ensuremath u}
\newcommand*{\mom}{\dens\vel}
\newcommand*{\toten}{\ensuremath E}
\newcommand*{\inten}{\ensuremath e}
\newcommand*{\press}{\ensuremath p}
\renewcommand*{\ss}{\ensuremath a}
\newcommand*{\jac}{\matr A}
%
\newcommand*{\abs}[1]{\lvert#1\rvert}
\newcommand*{\bigabs}[1]{\bigl\lvert#1\bigr\rvert}
\newcommand*{\biggabs}[1]{\biggl\lvert#1\biggr\rvert}
\newcommand*{\norm}[1]{\lVert#1\rVert}
%
\newcommand*{\e}[1]{\times 10^{#1}}
\newcommand*{\ex}[1]{\times 10^{#1}}%shorthand -- for use e.g. in tables
\newcommand*{\exi}[1]{10^{#1}}%shorthand -- for use e.g. in tables
\newcommand*{\nondim}[1]{\ensuremath{\mathit{#1}}}% italic iflg. ISO. (???)
\newcommand*{\rey}{\nondim{Re}}
\newcommand*{\acro}[1]{\textsc{\MakeLowercase{#1}}}%acronyms etc.

\newcommand{\nto}{\ensuremath{\mbox{N}_{\mbox{\scriptsize 2}}}}
\newcommand{\chfire}{\ensuremath{\mbox{CH}_{\mbox{\scriptsize 4}}}}
%\newcommand*{\checked}{\ding{51}}
\newcommand{\coto}{\ensuremath{\text{CO}_{\text{\scriptsize 2}}}}
\newcommand{\celsius}{\ensuremath{^\circ\text{C}}}
\newcommand{\clap}{Clapeyron~}
\newcommand{\subl}{\ensuremath{\text{sub}}}
\newcommand{\spec}{\text{spec}}
\newcommand{\sat}{\text{sat}}
\newcommand{\sol}{\text{sol}}
\newcommand{\liq}{\text{liq}}
\newcommand{\vap}{\text{vap}}
\newcommand{\amb}{\text{amb}}
\newcommand{\tr}{\text{tr}}
\newcommand{\cubic}{\text{cb}}
\newcommand{\crit}{\text{crit}}
\newcommand{\entr}{\ensuremath{\text{s}}}
\newcommand{\fus}{\text{fus}}
\newcommand{\flash}[1]{\ensuremath{#1\text{-flash}}}
\newcommand{\spce}[2]{\ensuremath{#1\, #2\text{ space}}}
\newcommand{\spanwagner}{\text{Span--Wagner}}
\newcommand{\triplepoint}{\text{TP triple point}}
\newcommand{\wrpt}{\text{with respect to~}}
\newcommand{\dd}[1]{\mathrm{d}{#1}}
\newcommand{\resid}{\text{res}}

\title{Modifying the cubic ecuation of state by a Pèneloux volume shift}
\author{Morten Hammer}
\graphicspath{{gfx/}}

\begin{document}
\maketitle
\section{Introduction}
Based on the article by P{\'e}neloux et al.\cite{Peneloux1982}, the
modification on the chemical potentials are investigated.

The volume shift introduced by P{\'e}neloux et al.,
\begin{equation}
  c = \frac{1}{n}\underset{i}{\sum}c_i n_i,
\label{eq:volumeshift}
\end{equation}
where $c_i$ is a component constant representing the component volume
shift.

In order to preserve the consistency of the cubic EoS, the same
correction is done on all phases. The effect of the shift for low
pressure gas is small, because here the specific volume is large. 

\section{Cubic equations of state}
The general cubic equation of state has the form
\begin{equation}
  P = \frac{RT}{v_\cubic -b_\cubic} - 
  \frac{\alpha a}{(v_\cubic-m_1 b_\cubic)(v_\cubic-m_2 b_\cubic)},
\label{eq:gencubic}
\end{equation}
where $m_1$ and $m_2$ are dimensionless constants defining the SRK, PR
and Van der Waals equation of state, $v=V/n~(\SI{}{\meter^3\per\mol})$
is the specific volume. The parameters $a~(\SI{}{\meter^3\joule\per
  \mol^2})$ and $b~(\SI{}{\meter^3\per\mol})$ typically depend on the
composition and the dimensionless quantity $\alpha$ is a function of
temperature.

\section{Cubic equations of state with volume shift}
The actual volume $v$ will be the cubic volume, $v_\cubic$, shifted by $c$,
\begin{equation}
  v = v_\cubic - c.
\label{eq:actvolume}
\end{equation}
In the same manner the co-volume paramater is shifted,
\begin{equation}
  b = b_\cubic - c.
\label{eq:covolume}
\end{equation}
That is, the actual eos to be solved is
\begin{align}
  P =& \frac{RT}{v - b} -
  \frac{\alpha a}{(v + c - m_1 (b + c))(v + c - m_2 (b + c))}\\
  = & \frac{RT}{v -b} - \frac{\alpha a}{\left( m_1-m_2 \right)(b + c)}
  \left(\frac{1}{v + c - m_1 (b + c)} - \frac{1}{v + c - m_2 (b +
      c)}\right)
\label{eq:cubicshifted}
\end{align}
There is no need to solve the shifted cubic eos, because
\begin{enumerate}
\item Given volume, the volume can be shifted before evaluating the
  cubic eos.
\item Given pressure, the resulting volume can be shifted.
\end{enumerate}
But the corrections for all the thermodynamic potentials and
properties must be known.

The fugacity coefficient is given by Pèneloux et al.,
\begin{equation}
  \ln \varphi_i = \ln \varphi_{\cubic,i} - \frac{c_i P}{R T}.
\label{eq:fugcoeff}
\end{equation}

\section{Required differentials}
Need to know how the Gibbs free energy, entropy, enthalpy,  and
internal energy is affected.

The residual Gibbs free energy, from definition and Equation
\eqref{eq:fugcoeff},
\begin{align}
  G^\resid =& R T \underset{i}{\sum}n_i \ln \varphi_i\\
  =& G^\resid_\cubic - nPc.
\label{eq:G}
\end{align}

The potentials only differ due to the shift in $Z$ due to the volume shift. 
The residual entropy, 
\begin{equation}
  S^\resid = -\pone{G^\resid}{T}{P,\vektor{n}} = S^\resid_\cubic .
\label{eq:S}
\end{equation}
is unchanged. 

The residual enthalpy,
\begin{equation}
  H^\resid = G^\resid + T S^\resid = H^\resid_\cubic -  nPc.
\label{eq:H}
\end{equation}
is affected in the same manner as the Gibbs free energy.

Internal energy,
\begin{equation}
  U^\resid = H^\resid - P V^\resid = U^\resid_\cubic.
\label{eq:U}
\end{equation}
is not affected.

The compressibillity factor become,
\begin{equation}
  Z = \frac{PV}{nRT} = Z_\cubic - \frac{nPc}{nRT}.
\label{eq:Z}
\end{equation}

The $nPc$ shift in enthalpy and Gibbs free energy must be
differentiated.
\begin{align}
  \pone{nPc}{T}{P,\vektor{n}} &= 0,\\ 
  \pone{nPc}{P}{T,\vektor{n}} &= nc,\\
  \pone{nPc}{n_i}{P,T,n_j} &= P c_i .
\label{eq:nPcdiff}
\end{align}
As it is seen these are straight forward.

Fugacity coefficient differentials:
\begin{align}
  \ln \varphi_{i,j} & = \ln \varphi_{\cubic,i,j},\\
  \ln \varphi_{iT} &= \ln \varphi_{\cubic,iT} + \frac{c_i P}{R T^2},\\
  \ln \varphi_{iP} &= \ln \varphi_{\cubic,iP} - \frac{c_i}{R T}.
\label{eq:fugdiff}
\end{align}

Volume differentials:
\begin{align}
  v_{i} & = v_{\cubic,i} - c_i.
\label{eq:voldiff}
\end{align}
The compressibillity factor differentials:
\begin{align}
  Z_{T} &= Z_{\cubic,T} + \frac{Pc}{RT^2},\\
  Z_{P} &= Z_{\cubic,P} - \frac{c}{RT},\\
  Z_{i} &= Z_{\cubic,i} - \frac{Pc_i}{nRT} + \frac{Pc}{nRT}.
\label{eq:Zdiff}
\end{align}
\section{Calculating the $c$}
The $c$ for the SRK EOS is calculated from the following equation:
\begin{equation}
  c_i = 0.40768\frac{R T_{c_i}}{P_{c_i}}\left(0.29441- Z_{\text{RA}}\right)
\label{eq:ci}
\end{equation}

$Z_{\text{RA}}$ are tabulated in TPlib. Reid et al. \cite{Reid1987}
also correlate $Z_{\text{RA}}$ as follows:
\begin{equation}
  Z_{\text{RA}} = 0.29056 - 0.08775 \omega
\label{eq:zra}
\end{equation}

Jhaveri and Youngren \cite{Jhaveri1988} have developed different paramaters for the PR EOS:
\begin{equation}
  c_i^{\text{PR}} = 0.50033\frac{R T_{c_i}}{P_{c_i}}\left(0.25969- Z_{\text{RA}}\right)
\label{eq:ci_PR}
\end{equation}
\clearpage
\bibliographystyle{plain}
\bibliography{../thermopack}

\end{document}
